\documentclass[a4paper, 11pt]{amsart}


% lesser margins
\usepackage{geometry}
\geometry{a4paper}
\geometry{twoside=false}

% no indent, but vertical spacing
\usepackage[parfill]{parskip}
\setlength{\marginparwidth}{2cm}

% clickable tocs
\usepackage{hyperref}

% floating figures
\usepackage{float}

\usepackage{tikz}
\usepackage{pgfplots}
\pgfplotsset{compat=newest}

\usepackage{graphicx}
\usepackage{caption}
\usepackage{subcaption}

% Matrices have a upper bound for its size
\setcounter{MaxMatrixCols}{20}

% Remove trailing `contents` after toc
\renewcommand{\contentsname}{}

\DeclareMathOperator{\diag}{diag}
\DeclareMathOperator{\logt}{\log_2}
% \newcommand{\vec}[1]{\mathbf{#1}}
\newcommand{\BigO}[1]{\mathcal{O}(#1)}
\newcommand{\proc}[1]{#1}

\newcommand{\todo}[1]{
	\addcontentsline{tdo}{todo}{\protect{#1}}
	$\ast$ \marginpar{\tiny $\ast$ #1}
}

\theoremstyle{plain}
\newtheorem{theorem}{Theorem}[section]
\newtheorem{lemma}[theorem]{Lemma}

\newcommand*{\thead}[1]{\multicolumn{1}{c}{\bfseries #1}}


\title{Parallel wavelet transform}
\author{Joshua Moerman}

\begin{document}

%%%%% INTRO IMAGE
% \tikzstyle{plain_line}=[]
% \begin{figure}
% 	\centering
% 	\begin{subfigure}[b]{0.5\textwidth}
% 		\begin{tikzpicture}
% 		\begin{groupplot}[group style={group size=1 by 4}, clip=false, yticklabels={,,}, height=3cm, width=\textwidth, xmin=0, xmax=128, ymin=-1, ymax=1, domain=0:128]
% 		\nextgroupplot
% 		\addplot[plain_line] coordinates {(0,0) (1,0) (1,1) (2,1) (2,0) (128,0)}; \legend{$e_1$}
% 		\nextgroupplot \addplot[plain_line] coordinates {(0,0) (2,0) (2,1) (3,1) (3,0) (128,0)}; \legend{$e_2$}
% 		\nextgroupplot \addplot[plain_line] coordinates {(0,0) (3,0) (3,1) (4,1) (4,0) (128,0)}; \legend{$e_3$}
% 		\nextgroupplot \addplot[plain_line] {0.8*sin(1*360*x/128) + 0.2*sin(3*360*x/128) + 0.08*sin(5*360*x/128)};
% 		\end{groupplot}
% 		\end{tikzpicture}
% 		\caption{Representing a signal on the standard basis.}
% 	\end{subfigure}~
% 	\begin{subfigure}[b]{0.5\textwidth}
% 		\begin{tikzpicture}
% 		\begin{groupplot}[group style={group size=1 by 4}, yticklabels={,,}, height=3cm, width=\textwidth, xmin=0, xmax=128, ymin=-1, ymax=1, domain=0:128]
% 		\nextgroupplot \addplot[plain_line] {sin(1*360*x/128)}; \legend{$f_1$}
% 		\nextgroupplot \addplot[plain_line] {sin(3*360*x/128)}; \legend{$f_3$}
% 		\nextgroupplot \addplot[plain_line] {sin(5*360*x/128)}; \legend{$f_5$}
% 		\nextgroupplot \addplot[plain_line] {0.8*sin(1*360*x/128) + 0.2*sin(3*360*x/128) + 0.08*sin(5*360*x/128)};
% 		\end{groupplot}
% 		\end{tikzpicture}
% 		\caption{Representing a signal on the Fourier basis.}
% 	\end{subfigure}
% 	\caption{We can represent the same signal on different basis. Note that the Fourier representation is smaller in this case.}
% 	\label{fig:basicplot}
% \end{figure}

% $$ 0.088 + 0.174 \times 0.257 $$
% $$ 0.798 \times 0.201 + 0.081 $$
% $$ = \ldots + $$


%%%%% ERROR IMAGE
\tikzstyle{plain_line}=[]
\begin{figure}
	\centering
	\begin{subfigure}[b]{0.5\textwidth}
		\begin{tikzpicture}
		\begin{groupplot}[group style={group size=1 by 1}, clip=false, yticklabels={,,}, height=3cm, width=\textwidth, xmin=0, xmax=128, ymin=-1, ymax=1, domain=0:128]
		\nextgroupplot \addplot[plain_line] {0.8*sin(1*360*x/128) + 0.2*sin(3*360*x/128) + 0.08*sin(5*360*x/128)};
		\end{groupplot}
		\end{tikzpicture}
		\caption{Representing a signal on the standard basis.}
	\end{subfigure}~
	\begin{subfigure}[b]{0.5\textwidth}
		\begin{tikzpicture}
		\begin{groupplot}[group style={group size=1 by 1}, yticklabels={,,}, height=3cm, width=\textwidth, xmin=0, xmax=128, ymin=-1, ymax=1, domain=0:128]
		\nextgroupplot \addplot[plain_line] {0.8*sin(1*360*x/128) + 0.2*sin(3*360*x/128) + 0.08*sin(5*360*x/128) + 0.5*sin(10*360*x/128)};
		\end{groupplot}
		\end{tikzpicture}
		\caption{Representing a signal on the Fourier basis.}
	\end{subfigure}
	\caption{We can represent the same signal on different basis. Note that the Fourier representation is smaller in this case.}
	\label{fig:basicplot}
\end{figure}

\end{document}
